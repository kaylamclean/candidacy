\startchapter{The Mono-$Z(\ell\ell)$ Search}
\label{chapter:prevWork}

% --------------------------------------------------------------------------------------
\section{Analysis Overview}

There are several important aspects of the mono-$Z(\ell\ell)$ analysis. Once data is collected and available to analyze, one of the the first steps is to optimize the event selection for the specific signal being searched for. For the mono-$Z(\ell\ell)$ search, events are selected in order to isolate two electrons or muons, with invariant mass close to the $Z$ mass, recoiling against a sizeable $E_{T}^{\text{miss}}$. The most important kinematic variables are chosen and are applied to the objects reconstructed from ATLAS data. Table \ref{tab:selections} summarizes the event selection requirements for the analysis of the 2015+2016 dataset.

\begin{table}[htbp]
\begin{center}
\begin{tabular}{c|c}
\hline
\hline
Object/Variable & Selection  \\ \hline
Lepton pair  &
Exactly one $e^+e^-$ or $\mu^+\mu^-$ pair with leading\\ & (subleading) lepton $p_T$ > 20 (30) GeV \\ \hline
Third lepton &  Veto any additional leptons with $\pt > $ 7 GeV\\ \hline
$m_{\ell\ell}$ & 76-106 GeV  \\ \hline
$E_{T}^{\text{miss}}$ & $> 90$ GeV \\ \hline
$\Delta R_{\ell\ell}$ & < 1.8  \\ \hline
$|\Delta\phi(p_T(Z),E_{T}^{\text{miss}})|$ & > 2.7  \\ \hline
Fractional $p_T$ difference &  < 0.2  \\ \hline
$E_{T}^{\text{miss}} / H_\text{T}$ & > 0.6 \\ \hline
b-jets & Veto b-tagged jets  \\ \hline \hline
\end{tabular}
\end{center}
\caption{Event selection cuts used for the mono-$Z(\ell\ell)$ analysis on the 2015+2016 dataset.}
\label{tab:selections}
\end{table}

- signal simulation\\
- major backgrounds (pie chart?)\\
- systematic errors\\
- limit setting on MET distribution\\

% --------------------------------------------------------------------------------------
\section{Truth Studies} 

- explain difference between reconstructed and truth samples\\
- motivation: at generator level we ignore detector effects; also sometimes cannot have access to many reconstructed samples, we can produce them ourselves\\
- MonoZTruthUVic framework for applying truth-level analysis cuts\\

% -------------------------------------------
\subsection{Theory Systematics on the Signal Acceptance}
- QCD renormalization/factorization scale systematics\\
- parton showering systematics\\
- mention possibility to use weights instead\\

% --------------------------------------------------------------------------------------
\section{Estimation of the $Z$+jets Background}

% -------------------------------------------
\subsection{ABCD Method}

- overview of the method\\
- challenges (correlations, low stats, large systematics)\\
- mention methodology used for EPS (?)\\

% -------------------------------------------
\subsection{$\gamma$+jets Technique}

- overview of the method\\
- challenges (trial and error with reweighting in 1D, 2D, or 2x1D; smearing, resolution of the Z better than the photon)\\

% --------------------------------------------------------------------------------------
\section{Dark Matter Limit Setting}

- some theory of hypothesis testing (binned likelihoods, p-values, discovery vs upper limits)\\
- MonoZLimitsUVic framework for setting limits\\
- show dmA and dmV results from ICHEP and EPS\\
- mention rescaled NLO limits for DM summary paper (?)\\
- show 2HDMa results for DM summary paper\\

% -------------------------------------------
\subsection{Mass Point Emulation}

- create a finer grid of points without using reconstructed samples\\
- studies done to show that it's possible to scale from dmA -> dmA in the on-shell region for a fixed mediator mass\\
- studies also done to show it's possible to scale dmA -> dmV\\

% --------------------------------------------------------------------------------------
\section{Analysis Software}
- mention efforts to update/improve our code (?)