\startfirstchapter{Introduction}
\label{chapter:introduction}

Over the past 10 months the focus of my M.Sc. project has been working on the UVic mono-$Z$ dark matter analysis. This work has primarily involved using Monte Carlo generators to simulate mono-$Z$ signal and background events. I have used \textsc{MadGraph} extensively to simulate the hard scatter of our primary background process, as well as for several mono-$Z$ dark matter models. I also learned how to use \textsc{Pythia} alongside \textsc{MadGraph} in order to simulate underlying and soft processes. These truth events were analyzed in ROOT in order to study their kinematic distributions, and I carried out various studies in order to confirm their validity. In addition, my most recent, ongoing contribution to the mono-$Z$ analysis involves characterizing the systematic errors on the signal acceptance due to renormalization and factorization scale uncertainties. A summary of my M.Sc. work is given in Chapter \ref{chapter:previous}. 

The proposed Ph.D. project is to continue work on the mono-$Z$ analysis with the 2015 dataset, with the goal to carry out a complete analysis with the 2015-17 Run 2 dataset of 100-150 fb$^{-1}$. The major tasks involved include event selection and optimization, background modelling, quantifying systematic uncertainties on signal and background, and either quantifying a discovery or setting limits. I will summarize these steps in \ref{chapter:phd}. In addition to the Run 2 mono-$Z$ analysis, one year would be spent at CERN to directly contribute to the running of the ATLAS liquid argon calorimeter. The details of the proposed Ph.D. project are discussed in Chapter \ref{chapter:phd}.