\startchapter{Conclusions}
\label{chapter:conclusions}

In this report I have presented a summary of my M.Sc. contributions to the current mono-$Z$ analysis. This work has primarily involved simulating signal and background samples with \textsc{MadGraph} and \textsc{Pythia} for truth analyses, including signal and background studies and characterizing uncertainties on the signal acceptance due to the renormalization and factorization scales. An outline for the proposed Ph.D. project has been given, including the major steps that are required for the mono-$Z$ analysis, with some reference to the steps taken in Run 1. Work will be done for the current 3.21 fb$^{-1}$ analysis before moving towards a larger dataset of $\sim$100 fb$^{-1}$ for the Ph.D. thesis. Continuing the mono-$Z$ analysis would be a natural step, given my contributions to the analysis so far. I have been an active part of the analysis since the beginning of my M.Sc. and I am keen to have a more critical role. This is an exciting time to be working in particle physics and I am eager to continue working on the mono-$Z$ analysis as a Ph.D. student. I look forward to learning more about dark matter and other ongoing searches, and also gaining a more intimate understanding of the operation of the LAr system and the whole of the ATLAS detector.