\subsection{Background modelling}
\label{sec:bgest}
	
\begin{table}[h]
\centering
\begin{tabular}{|c|c|c|}
\hline
Background 				& Description												& Estimation method \\ \hline \hline
$ZZ$					& real $Z$, real $E_\text{T}^\text{miss}$ from $Z$ decay to $\nu$'s		& Monte Carlo       \\ \hline
$WZ$					& real $Z$, real $E_\text{T}^\text{miss}$, lost lepton from $W$			& Monte Carlo       \\ \hline
$WW$					& two leptons, no $Z$-boson, real $E_\text{T}^\text{miss}$			& Data-driven       \\ \hline
top        					& two leptons, no $Z$-boson, real $E_\text{T}^\text{miss}$			& Data-driven       \\ \hline
$Z$+jets   				& real $Z$, fake $E_\text{T}^\text{miss}$                           				& Data-driven       \\ \hline
$W$+jets					& fake lepton, no $Z$-boson, real $E_\text{T}^\text{miss}$         		& Data-driven       \\ \hline
\end{tabular}
\caption[Mono-$Z$ background descriptions and estimation methods]{Mono-$Z$ background descriptions and estimation methods \cite{Barberio:1529950}.}
\label{tbl:backgrounds}
\end{table}

Perhaps the most involved task in the analysis is to estimate the relevant backgrounds. The backgrounds for the mono-$Z$ analysis, as studied in Run 1, are summarized in Table \ref{tbl:backgrounds}. There are several unique methods used for estimating each background, and it is preferable to use data-driven methods when possible. I will briefly summarize the steps taken in Run 1 \cite{Barberio:1529950}. These techniques will serve as a guide for the Run 2 analysis.

\begin{table}[h]
\centering
\begin{tabular}{|c|c|c|}
\hline
Process                           & 3-lepton & 4-lepton \\ \hline \hline
$ZZ \rightarrow \ell\ell\ell\ell$ & $RRRM$     & $RRRR$     \\ \hline
$WZ$                              & $RRR$      & $RRRF$     \\ \hline \hline
$t \bar{t}$                       & $RRF$      & $RRFF$     \\ \hline
$Z$+jets                          & $RRF$      & $RRFF$     \\ \hline
\end{tabular}
\caption[A summary of the processes contributing to the 3- and 4-lepton control regions used to scale the $ZZ$ and $WZ$ backgrounds]{A summary of the processes contributing to the 3- and 4-lepton control regions used to scale the $ZZ$ and $WZ$ backgrounds. $R$ indicates a real lepton, $F$ indicates a fake lepton, and $M$ indicates a missing lepton \cite{Barberio:1529950}.}
\label{tbl:zzwz}
\end{table}

The two dominant backgrounds in this analysis are the $ZZ$ and $WZ$ backgrounds. The $ZZ$ background is the largest and arises when one $Z$ decays to leptons and the other to neutrinos. The $WZ$ background arises when the $Z$ decays to leptons and the lepton from the $W$ is not reconstructed. These backgrounds are estimated using Monte Carlo. 3- and 4-lepton control regions are used to compare the $ZZ$ and $WZ$ MC estimates to data and determine a corresponding scale factor. The processes contributing to these control regions are summarized in Table \ref{tbl:zzwz}. $t\bar{t}$ and $Z$+jets processes also contribute to the control regions. The scale factor is then applied to the $ZZ$ and $WZ$ MC samples to obtain an estimate for the backgrounds in the signal region. In Run 1 the additional step of comparing the nominal \textsc{PowhegBox} $ZZ$ and $WZ$ samples to \textsc{Sherpa} $ZZ$ and $WZ$ samples to obtain systematic uncertainties on the MC is also taken.

The $WW$ and top backgrounds are merged into one background study that includes $WW$, $t\bar{t}$, $Wt$, and $Z \rightarrow \tau\tau$ contributions. These backgrounds all produce two same-flavour leptons and real $E_\text{T}^\text{miss}$ that comes from neutrinos. They are estimated using data in an $e\mu$ control region that is identical to the signal region except that it requires different-flavour leptons; in addition, this particular background study is performed using up to and including the $E_\text{T}^\text{miss}$ cut, in order to preserve statistics. Due to the flavour symmetry of the weak interaction, the ratios for the $WW$, $t\bar{t}$, $Wt$, and $Z \rightarrow \tau\tau$ processes to produce $ee$, $\mu\mu$, and $e\mu$/$\mu e$ are 1:1:2. Therefore, by estimating how many $e\mu$ events are in the control region, the number of same-flavour background events in the $ee$ and $\mu\mu$ signal regions can be obtained respectively using the following formulas \cite{Barberio:1529950}:

\begin{equation}
N_{ee}^\text{est}\text{(bkg)} = \frac{1}{2} \times \epsilon \times N_{e\mu}^\text{data,corr}
\end{equation}

\begin{equation}
N_{\mu\mu}^\text{est}\text{(bkg)} = \frac{1}{2} \times \frac{1}{\epsilon} \times N_{e\mu}^\text{data,corr}
\end{equation}

\noindent $\epsilon$ accounts for the different detection efficiencies between electrons and muons in the $Z$-mass window, obtained from data. The number of events detected is proportional to the square of the efficiency: $N_{ee}^\text{data} \propto \epsilon_e^2$, and similarly $N_{\mu\mu}^\text{data} \propto \epsilon_\mu^2$. Then $\epsilon$ is the ratio of efficiencies, $\epsilon = \epsilon_e/\epsilon_\mu = \sqrt{N_{ee}^\text{data}/N_{\mu\mu}^\text{data}}$. $N_{e\mu}^\text{data,corr}$ is the number of $e\mu$/$\mu e$ background events in the control region from the four contributions of interest. Other processes contribute to this control region (including $WZ$) and must be subtracted off (hence the ``corr'' subscript for ``corrected''). These other contributions are estimated from MC and are shown to be less than $3\%$. Systematic uncertainties from this method come from comparison to MC samples, the difference in $\epsilon$ as estimated from MC compared to data, and from errors on the MC background samples used for the background subtraction.

The $Z$+jets background arises when the $Z$ decays to two leptons but the $E_\text{T}^\text{miss}$ is mis-measured. Because the $E_\text{T}^\text{miss}$ is non-physical, $\Delta \phi (p_\text{T}^{\ell\ell}, E_\text{T}^{\text{miss}})$ will be more diffuse when coming from $Z$+jets. There are two data-driven estimation methods used in Run 1. The first method involves making a fit to the $\Delta \phi (p_\text{T}^{\ell\ell}, E_\text{T}^{\text{miss}})$ distribution in the signal region to determine the contribution from $Z$+jets. To find the background contribution, a fit is made to $\Delta \phi (p_\text{T}^{\ell\ell}, E_\text{T}^{\text{miss}})$ in a low $E_\text{T}^\text{miss}$ (< 80 GeV) control region, where $Z$+jets dominates. The variations of the fit parameters with $E_\text{T}^\text{miss}$ are also modelled and incorporated into the fit. From this fit in the control region, the distribution is extrapolated into the higher $E_\text{T}^\text{miss}$ signal region, and the contribution from $Z$+jets can be determined. The fit is also done on MC in order to validate the method and obtain systematic uncertainties. The second method uses the ABCD method to estimate the $Z$+jets background using two independent kinematic distributions to distinguish between signal and background. Due to the unphysical properties of the fake $E_\text{T}^{\text{miss}}$ for this background, it is not well-correlated with the other kinematic variables that are associated with the $Z$, including $\Delta \phi (p_\text{T}^{\ell\ell}, E_\text{T}^{\text{miss}})$ and other angular variables. A 2-dimensional distribution of data is made for $E_\text{T}^{\text{miss}}$ vs an uncorrelated variable. $\eta^{\ell\ell}$, the pseudorapidity of the reconstructed $Z$, is used in Run 1 because of its low correlation of $1.3\%$. The 2D distribution is split into four regions: one signal region ($A$) with high $E_\text{T}^{\text{miss}}$ and small $\eta^{\ell\ell}$, and three sideband control regions ($B$, $C$, and $D$). The approximation used is that the ratio of the number of background events in $A$ compared to $B$ is the same as the ratio in $C$ compared to $D$. The number of background events in the signal region is then calculated using \cite{Barberio:1529950}

\begin{equation}
n(A) = \frac{n(B) \times n(C) \times \alpha}{n(D)}.
\end{equation}

\noindent $\alpha$ is the purity factor (amount of $Z$+jets compared to other backgrounds) estimated from MC. It is approximately 1 if only $Z$+jets dominates. Figure \ref{fig:abcd} shows the ABCD regions used in Run 1. The analysis is performed after a set of cuts that are similar to the signal region. A lower cut on $E_\text{T}^{\text{miss}}$ is required in order to be in a region where $Z$+jets dominates. The difference in the $Z$+jets background obtained using the fitting method vs the ABCD method is interpreted as a systematic error.

\begin{figure}[ht]
\centering
\includegraphics[width=0.7\textwidth]{Figures/ABCD2.pdf}
\caption[ABCD method used in Run 1]{ABCD method used in Run 1. $E_\text{T}^\text{miss}$ is plotted against $\eta^{\ell\ell}$ in the $\mu\mu$ channel for 20.3 fb$^{-1}$ of data. The event counts are given for each control region and the estimated count is given for the signal region (A).}
\label{fig:abcd}
\end{figure}

Finally, the $W$+jets background involves one jet being misidentified as a lepton, and so two leptons are reconstructed into a fake $Z$. This background will mostly affect the electron signal region, as a jet is 10 times more likely to fake an electron compared to a muon \cite{Barberio:1529950}. The fake $Z$ gives good discrimination power because the fake $Z$ mass and the angle between the leptons will be more diffuse. The $E_\text{T}^\text{miss}$ distribution for this background peaks near half of the $W$ mass, and so the statistics are very low at higher $E_\text{T}^\text{miss}$. For this reason, estimates for this background are made in a low $E_\text{T}^\text{miss}$ region and are then extrapolated to the signal region. $W$+jets contributions are estimated by loosening the selection and isolation requirements on one of the electrons so that there is a higher probability of it actually being a jet. The rest of the selections are applied up to and including the $Z$ window cut. Identical steps are taken on MC to obtain a scale factor for the data. Two different fits are then applied to the $E_\text{T}^\text{miss}$ distribution in different energy ranges. The fits are then assessed using a $\chi^2$ check. The best function is integrated from the signal region $E_\text{T}^\text{miss}$ cut up to 1 TeV, past which point the $W$+jets contribution is negligible. This gives an estimate for the number of background events in the signal region. In Run 1, the only signal region to have non-negligible $W$+jets contributions is the signal region with the lowest $E_\text{T}^\text{miss}$ cut of 150 GeV. The dominant systematics for this background are due to uncertainties that a jet will fake a lepton.
