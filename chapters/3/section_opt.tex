\subsection{Event selection}
\label{sec:opt}

There are several kinematic variables to place cuts on in order to isolate mono-$Z$ signal events. The selections currently being used in the analysis are shown in Table \ref{tbl:cutflow}.

\begin{table}[ht]
\centering
\begin{tabular}{|c||c|}
\hline
\# & Signal region cutflow											\\ \hline \hline
1 & 2 same-flavour leptons										\\ \hline 
2 & $p_{\text{T}_1}$ > 30 GeV										\\  \hline
3 & third lepton veto												\\  \hline
4 & opposite-sign leptons											\\  \hline
5 & 76 GeV < $m_{\ell\ell}$ < 106 GeV								\\  \hline
6 & $E_\text{T}^{\text{miss}}$ > 90 GeV								\\ \hline
7 & $\Delta \phi_{\ell\ell}$ < 1.7										\\  \hline
8 & $\Delta \phi (p_\text{T}^{\ell\ell}, E_\text{T}^{\text{miss}})$ > 2.6			\\  \hline
9 & $|p_\text{T}^{\ell\ell} - E_\text{T}^{\text{miss}}|/p_\text{T}^{\ell\ell}$ < 0.2	\\  \hline
10 & jet veto 													\\ \hline
\end{tabular}
\caption[Preliminary mono-$Z$ signal region selections]{Preliminary mono-$Z$ signal region selections.}
\label{tbl:cutflow}
\end{table}

\noindent Selections 1-5 and 10 are motivated primarily to reduce backgrounds and obtain events with a high quality lepton pair. Selections 6-9 are specifically meant to isolate mono-$Z$ dark matter events. The dark matter particles will be measured as $E_\text{T}^{\text{miss}}$ in mono-$Z$ events. The $Z$ boson will also be boosted if it recoils against a heavy mediator particle, and so the two leptons are expected to be collimated with a small $\Delta \phi_{\ell\ell}$. We also require that $p_\text{T}^{\ell\ell}$ and $E_\text{T}^{\text{miss}}$ should be back-to-back, so that $\Delta \phi (p_\text{T}^{\ell\ell}, E_\text{T}^{\text{miss}})$ is close to $\pi$. In addition, the $p_\text{T}^{\ell\ell}$ should balance with the $E_\text{T}^{\text{miss}}$, and so a small $|p_\text{T}^{\ell\ell} - E_\text{T}^{\text{miss}}|/p_\text{T}^{\ell\ell}$ is expected. 

The cuts that are specific to the mono-$Z$ analysis need to be optimized in order to maximize the amount of signal significance attainable. This is currently being done on the 2015 dataset by maximizing the significance \cite{Cowan:2010js}:

\begin{equation}
S = \sqrt{2 \left( \left( s+b \right) \ln \left( 1+\frac{s}{b} \right) -s \right)}
\label{eqn:s}
\end{equation}

\noindent For a given model and signal region ($ee$ or $\mu\mu$), the sensitivity can be calculated for several cut values on each of the optimization variables (variables 6-9). In Run 1, it was found that the significance was most sensitive to the cut on $E_\text{T}^{\text{miss}}$. Figure \ref{fig:sig} shows a plot of $E_\text{T}^{\text{miss}}$ for a particular dark matter model with backgrounds overlaid, and with all selections applied, for the current analysis. Because the $E_\text{T}^{\text{miss}}$ distribution varies greatly for different dark matter models, four different signal regions were chosen for this analysis in Run 1 with different $E_\text{T}^{\text{miss}}$ cuts. Models were then assigned a signal region according to their $E_\text{T}^{\text{miss}}$ distributions. Although only one signal region has been defined in the current analysis so far, it is very likely that multiple signal regions will be needed.

\begin{figure}[t]
\centering
\includegraphics[width=0.7\textwidth]{Figures/sig.pdf}
\caption[$E_\text{T}^{\text{miss}}$ distribution for vector-mediated dark matter sample with $m_\text{DM} = 50$ GeV and $m_\text{med} = 95$ GeV and backgrounds overlaid]{$E_\text{T}^{\text{miss}}$ distribution for vector-mediated dark matter sample with $m_\text{DM} = 50$ GeV and $m_\text{med} = 95$ GeV and backgrounds overlaid.}
\label{fig:sig}
\end{figure}

Optimization is underway for the current analysis using the latest Moriond 2016 recommendations, and similar steps will need to be done for a new, larger dataset. The analysis will become sensitive to more models as more data is collected, and so the optimization will need to be done again to adjust to the set of the most important models.