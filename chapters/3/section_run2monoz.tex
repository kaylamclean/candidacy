\section{Run 2 mono-$Z$ analysis}
\label{sec:monoZ}

The current proposal for the Ph.D. project is to continue the ongoing mono-$Z$ analysis being done at the University of Victoria with the 2015 dataset of 3.21 fb$^{-1}$. The analysis will then be extended to $\sim$100 fb$^{-1}$ of data, to be collected by the end of 2017, for the Ph.D. thesis.

The mono-$Z$ dark matter search is complementary to similar mono-photon, mono-jet, and mono-$W$ searches. Tagging on a $Z$ boson reduces fake QCD backgrounds after applying a $Z$ mass window cut. The mono-$Z$ channel also has unique $ZZ\chi\chi$ and $\gamma^*Z\chi\chi$ couplings that are unavailable in other mono-$X$ searches. 

In Run 1, the main focus of the mono-$Z$ analysis was on effective field theory (EFT) models. These models assume a very heavy mediating particle compared to the energy transfer of the process. The validity of these approximations have been brought into question as Run 2 has increased to a 13 TeV centre-of-mass energy. Hence, Run 2 is largely focused on testing so-called simplified models, where the $Z$ boson is emitted by a quark as initial-state radiation (ISR) and a mediator particle couples the quarks to the dark matter WIMP pair in either $s$-channel or $t$-channel type processes. An effective field theory model is also studied with the aforementioned $ZZ\chi\chi$ and $\gamma^*Z\chi\chi$ couplings. The Feynman diagrams for these models are given in Appendix \ref{chapter:appendix}.

In the following sections I will discuss some of the major tasks required for the mono-$Z$ analysis, with reference to the Run 1 paper \cite{Aad:2014vka} and corresponding support note \cite{Barberio:1529950}.