\subsection{Systematic uncertainties}
\label{sec:syst}

In the previous section I discussed some of the systematic uncertainties associated with background modelling; each method used to estimate mono-$Z$ backgrounds will have inherent systematic errors. 

Systematic uncertainties also arise due to detector performance, object reconstruction, and object identification algorithms. These uncertainties affect all background and signal Monte Carlo samples. Electrons have systematic uncertainties due to errors in the reconstruction and identification efficiencies, as well as energy resolution and energy scale uncertainties. Muons are similar and have reconstruction efficiency systematics, $p_\text{T}$ resolution systematics, and energy scale uncertainty. Jets also have inherent errors due to uncertainties in their energy scale and resolution, and missing transverse momentum measurements from the calorimeter have corresponding systematics. There are several \texttt{RootCore} combined performance tools available to automatically apply $\pm 1 \sigma$ variations to the objects of interest in order to observe variations in the yields. These tools change frequently according to the latest combined performance recommendations and their implementation must be kept updated in the analysis code.

There are also theoretical systematic uncertainties due to QCD scale uncertainties, as discussed earlier, as well as PDF uncertainties. To characterize the PDF uncertainties, a reweighting scheme is used to reweight on an event-by-event basis according to a set of variational PDFs. The set of variational PDFs are generated by varying each PDF fit parameter by $\pm 1\sigma$. These types of errors are known as intra-PDF uncertainties. Inter-PDF uncertainties are characterized by studying the effects due to using a completely different PDF set. These PDF uncertainties are currently being studied on the 2015 dataset. Similar analyses for the QCD and PDF uncertainties will need to be done for the 100 fb$^{-1}$ dataset. We are also investigating if parton showering systematics will be applicable. There are also additional systematics due to errors in the luminosity, pileup, etc. that need to be applied as well. 