\chapter{Analysis with the Full Run 2 Dataset}
\label{chapter:fullRun2}

% --------------------------------------------------------------------------------------
\section{Event Selection Optimization}

For the full Run 2 analysis the event selection for the signal region will need to be reoptimized. This is typically done by quantifying the signal significance. 

- different MET working points\\
% https://twiki.cern.ch/twiki/bin/view/AtlasProtected/EtmissRecommendationsRel21p2#Working_Points

- MET significance \\ 
%https://cds.cern.ch/record/2294922/files/ATL-COM-PHYS-2017-1735.pdf

- signal/background or significance optimization, ...\\
% Chris's talk with 2 signif formulae: https://indico.cern.ch/event/678491/contributions/2856994/attachments/1584711/2505221/SignificanceCalculation.pdf
% Signif using formula from Cowan paper: https://indico.cern.ch/event/678491/contributions/2895815/attachments/1600870/2538022/WhitePaperFinalization.pdf
% Cowan paper with 2 formulae with likelihoods: https://www.pp.rhul.ac.uk/~cowan/stat/medsig/medsigNote.pdf

% --------------------------------------------------------------------------------------
\section{Applying the \gjets Method to Data}

- continue efforts with the \gjets method\\
- try MET cut before reweighting?\\
- obtain weights from MC?\\
- define a validation region\\
- apply weights to data, compare with MC to test closure\\
- implementation in MonoZUVic\\

% --------------------------------------------------------------------------------------
\section{Signal Models}
- s-channel simplified models at NLO\\
% DM summary paper? https://cds.cern.ch/record/2273840/files/ATL-COM-PHYS-2017-1031.pdf
- mention rescaling to NLO and with leptonic couplings\\

- t-channel signals: Bell model (mediator-Z diagram), less simplified model\\

It should also be noted that, in the time since these initial benchmark models were set, next-to-leading order (NLO) models are now being considered instead of the leading order (LO) tree diagrams discussed here. Finally, in addition to these $s$-channel diagrams there also exist $t$-channel processes that are of great interest to the \monoZ search. These will be discussed more in Chapter \ref{chapter:fullRun2}.

% --------------------------------------------------------------------------------------
\section{Prospective Limits with 140 \ifb}

\begin{figure}[htb]
\centering
\includegraphics[width=0.7\textwidth]{Figures/140ifb.png}
\caption{Prospective vector exclusion limit with 140 \ifb. Produced by Chris Anelli.}
\label{fig:id}
\end{figure}

% --------------------------------------------------------------------------------------
\section{Other Analysis Improvements}

- comparison with ID measurements\\

\begin{figure}[htb]
\centering
\includegraphics[width=0.5\textwidth]{Figures/id.png}
\caption{A schematic of what a comparison with ID detection measurements could look like.}
\label{fig:id}
\end{figure}

- theoretical uncertainties on signal acceptance - weights!! \\
- maintenance of analysis code\\