\startchapter{Conclusions}
\label{chapter:conclusions}

An overview of the \monoZ search has been presented, including the progress made so far and the work to be done for the full Run 2 analysis. An introduction to the LHC and ATLAS detector is given alongside the theory of dark matter searches at colliders. So far, the \monoZ analysis has set limits on the benchmark $s$-channel simplified models and 2HDM+PS models. The analysis involves several aspects, including event selection optimization, background estimation, systematic evaluation, and setting exclusion limits. Previous contributions in these categories are discussed, including the estimation of the \Zjets background using two different techniques (the ABCD method and the developing \gjets reweighting scheme), estimation of theoretical uncertainties on the dark matter signal acceptance, and limit setting for the simplified and 2HDM+PS models. There are several improvements to be done for the full Run 2 dataset, projected to contain 140 \ifb of data. The most important improvement will be the addition of new models to the analysis, including $t$-channel processes unique to the \monoZ search, and the continued pursuit of the 2HDM+PS models. The \gjets estimation for the \Zjets background will continue to be developed and finalized, and new variables such as the \etmiss significance and particle-flow \etmiss will be studied, hopefully leading to improvements in the primary observable of the analysis. Other reinterpretations of the \monoZ exclusion limits will also be investigated. With 140 \ifb of data and a variety of models to be studied, the \monoZ search looks promising.