\startchapter{Dark Matter Searches with the ATLAS Detector}
\label{chapter:theory}

% --------------------------------------------------------------------------------------
\section{Dark Matter Theory}

Effective field theories (EFTs) \cite{Boveia:2016mrp} \cite{Beltran:2010ww} were the primary models studied in \etmissX dark matter searches in Run 1, when the centre-of-mass energy was 8 TeV. In short, these theories assume that a dark matter pair is produced by means of a contact interaction with a quark and antiquark, as illustrated in Figure \ref{fig:eft}. These types of models offer a straightforward means to compare collider results to DD or ID experiments. However, an important caveat of EFTs is that they are only valid when the mass of the mediating particle between the \chichi and \qq is much heavier than the momentum transfer of the process. Now that the centre-of-mass energy has increased to 13 TeV in Run 2, these EFTs are no longer valid. Thus, a new baseline model is used in Run 2 with the mediator particle explicitly included.

\begin{figure}[htb]
    \centering
    \begin{subfigure}[b]{0.35\textwidth}
        \includegraphics[width=\textwidth]{Figures/eft1.png}
        \label{fig:eft1}
    \end{subfigure}
    ~ %add desired spacing between images, e. g. ~, \quad, \qquad, \hfill etc. 
      %(or a blank line to force the subfigure onto a new line)
    \begin{subfigure}[b]{0.35\textwidth}
        \includegraphics[width=\textwidth]{Figures/eft2.png}
        \label{fig:eft2}
    \end{subfigure}
    \caption{Representative EFT diagrams for the \monoZ signature \cite{Beltran:2010ww}.}
\label{fig:efts}
\end{figure}

Leading order simplified models are the first set of benchmark models used for \etmissX searches in Run 2, as recommended by the LHC DM Working Group \cite{Boveia:2016mrp}. An example $s$-channel diagram for the \Zetmiss signal is shown in Figure \ref{fig:simp}. These models are considered `simplified' because they introduce the minimum number of parameters needed to include a mediator between SM and dark matter particles (compared to more complicated models such as supersymmetry). 

\begin{figure}[htb]
\centering
\includegraphics[width=0.3\textwidth]{Figures/simp.png}
\caption{Simplified model $s$-channel diagram for the \monoZ signature. \cite{Aaboud:2017bja}}
\label{fig:simp}
\end{figure}

Simplified models introduce five new parameters: the mass of the WIMP, \mchi, the mass of the mediating particle, \mmed, the couplings of the mediator to the SM (to dark matter), \gq (\gchi), and the width of the mediator \Wmed. The mediator particle can be spin-0 (scalar or pseudo-scalar) or spin-1 (vector or axial-vector). 

%The interaction Lagrangians for models with vector and axial-vector mediator couplings are given in Equations \ref{eqn:Lvector}, and \ref{eqn:Laxial}:

%\begin{equation}
%\mathcal{L}_\text{vector} = g_q \sum_q \eta_\mu \bar{q} \gamma^\mu q + g_\chi \eta_\mu \bar{\chi} \gamma^\mu \chi
%\label{eqn:Lvector}
%\end{equation}

%\begin{equation}
%\mathcal{L}_\text{axial-vector} = g_q \sum_q \eta_\mu \bar{q} \gamma^\mu \gamma^5 q + g_\chi \eta_\mu \bar{\chi} \gamma^\mu \gamma^5\chi
%\label{eqn:Laxial}
%\end{equation}

Following the recommendations in Ref. \cite{Boveia:2016mrp}, for spin-0 models the couplings are set to \gq = \gchi = 1.0. The Yukawa couplings are also included between the quarks and the mediator. For spin-1 models the couplings are fixed to $g_q = 0.25$ and $g_\chi = 1.0$. In addition, assuming that the mediator has no additional decay modes, \Wmed is set to the so-called minimal width \cite{Abercrombie:2015wmb}, which is fixed by \gq, \gchi, \mchi, and \mmed. The couplings were chosen to correspond with a rough estimate of the lower sensitivity of the Run 2 mono-jet analysis, and so that \Wmed/\mmed$ < \sim 0.05$.

%Similarly, the interaction Lagrangians for models with scalar and pseudo-scalar mediator couplings are given by Equations \ref{eqn:Lscalar} and \ref{eqn:Lpseudo}:

%\begin{equation}
%\mathcal{L}_\text{scalar} = g_\chi \eta \bar{\chi} \chi + \frac{\eta}{\sqrt{2}} g_q \sum_i \left( y_i^u \bar{u}_i u_i + y_i^d \bar{d}_i d_i + y_i^\ell \bar{\ell}_i \ell_i \right)
%\label{eqn:Lscalar}
%\end{equation}

%\begin{equation}
%\mathcal{L}_\text{pseudo-scalar} = i g_\chi \eta \bar{\chi} \gamma_5 \chi + \frac{i \eta}{\sqrt{2}} g_q \sum_i \left( y_i^u \bar{u}_i \gamma_5 u_i + y_i^d \bar{d}_i \gamma_5 d_i + y_i^\ell \bar{\ell}_i \gamma_5 \ell_i \right)
%\label{eqn:Lpseudo}
%\end{equation}

% Assuming minimal flavour violation, spin-0 resonances will behave similarly to the Higgs boson (hence the Yukawa couplings)

Run 2 analyses have adopted the $s$-channel exchange of an axial-vector mediator as the primary benchmark scenario. This choice is motivated by the findings in Ref. \cite{Boveia:2016mrp} that show that collider searches can be more sensitive than DD experiments at low values of \mchi for this type of mediator. 

Although they have advantages compared to EFTs, simplified models are not a complete theory and violate unitarity for some regions of parameter space. At the beginning of Run 2 they were exceedingly useful in providing a guideline for the ATLAS and CMS collaborations to follow in tandem, but there is now a big push towards studying more theoretically complete models. This includes two Higgs doublet models, discussed next, as well as $t$-channel signatures with a coloured scalar mediator and so-called Less Simplified models, discussed in Chapter \ref{chapter:fullRun2}.

% https://arxiv.org/pdf/1701.07427.pdf
% DM summary paper: https://cds.cern.ch/record/2273840/files/ATL-COM-PHYS-2017-1031.pdf

A model that is becoming popular in Run 2 \monoX dark matter searches is known as the two Higgs doublet + pseudo-scalar (2HDM+PS) model \cite{Bauer:2017ota}. 2HDMs are essential for many well-motivated BSM theories. They are perturbative and avoid violating unitarity by allowing mixing between the dark matter mediator and other bosons. 

\begin{figure}[htb]
\centering
\includegraphics[width=0.75\textwidth]{Figures/2hdma.png}
% EPS paper: https://arxiv.org/abs/1708.09624
\caption{Representative 2HDM+PS diagrams for the \monoZ signature. The second diagram can have $A$ instead of $a$.}
\label{fig:2hdma}
\end{figure}

Figure \ref{fig:2hdma} shows the two main diagrams of the 2HDM+PS model with the \monoZ signature. This model introduces two CP-even scalars $H$ and $h$ (where $h$ is the SM Higgs), one CP-odd pseudo-scalar $A$, two charged scalars $H^+$ and $H^-$, and the pseudo-scalar $a$ that couples the SM particles to dark matter. There are also the parameters $\sin(\theta)$ and $\tan(\beta)$.

{\color{red}TODO: Discuss parameters, mono-Z sensitivity vs mono-H, more sensitive than mono-jet, ...}

\clearpage

% --------------------------------------------------------------------------------------
\section{The LHC and the ATLAS Detector}

The Large Hadron Collider (LHC) is the world's largest particle accelerator with a circumference of 27 km. Superconducting magnets are used to accelerate two beams of protons up to nearly the speed of light. The beams are then brought to collision at various points around the LHC. Located at one of these collision points is the ATLAS detector, one of the two multipurpose detectors at the LHC. The LHC has been colliding protons at a centre-of-mass (COM) energy of 13 TeV since 2015. During this time ATLAS has been continuously taking data.

The amount of \pp collision data delivered by the LHC is quantified by the \textit{luminosity}. The total number of $pp$ collisions $N$ detected over all time $t$ is related to the cross section for $pp$ collisions $\sigma$, and can be expressed in terms of either the instantaneous luminosity $L$ or the integrated luminosity $\mathcal{L}$:

\begin{equation}
N = \sigma = \int{L} \text{d}t = \sigma \mathcal{L}
\end{equation}

\noindent $\mathcal{L}$ is the measure of total data collected that is frequently quoted in ATLAS. It has units of cm$^2$, but a more frequently used unit is the inverse barn. 1 b = 10$^{-28}$ m$^2$. The total amount of data delivered by the LHC since 2015 is currently 93 \ifb from 2015-2017. ATLAS has recorded a total of 86 \ifb, with 80 \ifb that is good for physics analyses.

\begin{figure}[htb]
\centering
\includegraphics[width=1\textwidth]{Figures/atlas.jpg}
\caption{The ATLAS detector.}
\label{fig:atlas}
\end{figure}

An overview of the ATLAS detector is shown in Figure \ref{fig:atlas}. It is composed of four major subsystems. The innermost system is the inner detector (ID) which measures the tracks of charged particles very near to the collision point. It consists of three layers known as the pixel detector, semiconductor tracker (SCT), and transition radiation tracker (TRT). The innermost pixel detector has the highest resolution granularity in the detector and consists of 80 million pixels. The SCT consists of 60 m$^2$ of silicon microstrips with densely packed readout channels, and the TRT consists of 300,000 straw tubes with wires inside to measure tracks from ionization. The ID is encased in a solenoid magnet that exerts a 2 Tesla magnetic field. The magnetic field causes the paths of charged particles to bend. The momentum of the particles can be determined from the curvature of the tracks.

Moving outward from the centre of the detector, the next subsystems are the electromagnetic (EM) and hadronic (HAD) calorimeters. The EM system is entirely composed of liquid argon (LAr) calorimetery, while the HAD system includes the tile calorimeter in the barrel region and LAr calorimetery in the end caps. The calorimeters are dense and designed to stop particles completely so that their energy is deposited entirely inside the detector. The EM calorimeter is designed to stop particles that interact electromagnetically (electrons and photons) while the HAD calorimeter is designed to stop hadrons (e.g. protons and neutrons). The LAr calorimeter consists of alternating layers of copper absorber material and LAr ionization chambers, while the tile calorimeter alternates between layers of steel and plastic scintillators. 

The outermost and largest system of the detector is the muon spectrometer (MS). Muons will interact minimally with the detector and so the MS is designed to measure their momenta from tracks. The MS consists of several systems. Monitored drift tubes (MDTs) are used in the barrel and end caps for measuring track curvature. Resistive plate chambers (RPCs), cathode strip chambers (CSCs), and thin gap chambers (TGCs) are used for precision coordinate measurements throughout the spectrometer. The MS also contains the ATLAS toroid magnet system.

Events that occur in the detector are recorded by the ATLAS trigger system, consisting of tiers that act at the hardware and software levels. The trigger makes decisions on which events to record to disk or not based on energy deposits, etc. {\color{red}TODO!}

\clearpage

