\section{\textsc{MadGraph} and $ZZ$ background studies}
\label{sec:mg}

There are a variety of MC generators available for simulating the experimental outcomes of high-energy particle collisions. \textsc{MadGraph 5} is commonly used in ATLAS for the generation of few-body hard scatter processes. It is a very generic generator that is capable of simulating any processes for which the model has a well-defined Lagrangian, including exotic models like SUSY. The model is defined in \textsc{FeynRules}, a \textsc{Mathematica} package, before being imported into \textsc{MadGraph}. Arbitrary processes can be generated up to next-to-leading order in QED and QCD, depending on the model.

The motivation for learning how to use \textsc{MadGraph} is that ATLAS uses \textsc{MadGraph} to generate mono-$Z$ dark matter processes. These include simplified models with a mediator particle and WIMP dark matter. Of particular interest to the mono-$Z$ analysis for Run 2 are $s$-channel simplified models where the mediator can be a vector, scalar, pseudo-vector, or pseudo-scalar, as well as $t$-channel models with a scalar mediator. In addition, we consider an effective field theory model where the $Z$ couples directly to the dark matter particle via a contact interaction. The Feynman diagrams for these processes can be found in Appendix \ref{chapter:appendix}. These models are all simulated using \textsc{MadGraph}. I will discuss dark matter simulation in more detail in the next section.

\textsc{MadGraph} gives the user a lot of power in selecting the exact processes to be generated. There are difference ``cards" available where the user can set different parameters. For example, there is a run card for setting inputs such as the beam energies, PDF, scale factors, and cuts, as well as a parameter card for setting particle masses, widths, and decays/branching ratios. There are also special syntaxes for requiring $s$-channel processes, forbidding on-shell $s$-channel resonances, and excluding intermediate particles. \textsc{MadGraph} also automatically produces all Feynman diagrams for the process being simulated.

My initial work with \textsc{MadGraph} involved simulating $pp \rightarrow 2e2\nu$ and $pp \rightarrow ZZ \rightarrow 2e2\nu$ events, the major background of the mono-$Z$ analysis. Although POWHEG is the generator used in ATLAS to simulate this process, studying this Standard Model process with \textsc{MadGraph} was a good introduction for me to learn how to use the program, with the intention of using it later on to study dark matter. Throughout my excursions with \textsc{MadGraph} I also learned how to analyze $ZZ$ events, using ROOT to program a truth analysis. This was my first experience in creating a logical analysis program that selected particles, applied selections, and produced kinematic plots. I also took steps to validate my event generation, one of which included calculating cross sections for total $ZZ$ production and comparing against other theoretical predictions for different energies and orders. This gave me confidence in how I was using \textsc{MadGraph}.

\begin{figure}[hb]
\centering
\includegraphics[width=1\textwidth]{Figures/ppeevv_contributions.pdf}
\caption[$M_{ee}$ distributions for various $\gamma$, $W$, and $Z$ diagram contributions for $pp \rightarrow 2e2\nu$ events (unweighted) at NLO from \textsc{MadGraph}]{$M_{ee}$ distributions for various $\gamma$, $W$, and $Z$ diagram contributions for $pp \rightarrow 2e2\nu$ events (unweighted) at NLO from \textsc{MadGraph}. A linear plot is shown on the left and the logarithmic plot is shown on the right.}
\label{fig:ppeevv_contributions}
\end{figure}

\textsc{MadGraph} is also capable of handling interference effects. During my time learning \textsc{MadGraph}, some confusion arose regarding the cross section quoted in the AMI database for the $pp \rightarrow ZZ \rightarrow 2e2\nu$ process. The cross section was not compatible with the cross section given for $pp \rightarrow ZZ \rightarrow 4e$ using the narrow-width approximation and a back-of-the-envelope calculation using the $ZZ$ branching ratios; it was thought that only $Z$ contributions were included in the ATLAS sample (no $W$ or $\gamma$), which was produced using the POWHEG generator. Using \textsc{MadGraph}, I was able to generate $pp \rightarrow 2e2\nu$ events while including or excluding the $W$ and $\gamma$ contributions. Figure \ref{fig:ppeevv_contributions} shows the $M_{ee}$ distributions I obtained at NLO while neglecting a given boson. Although the black line is the only distribution that is physical, the other distributions show the interesting interference effects at play (for example, removing the photon contribution raises the distribution at the high-mass tail). From these samples, I found that the cross section for $pp \rightarrow 2e2\nu$ with no $W$ contribution very closely matched the quoted AMI cross section ($313 \pm 1$ fb obtained using \textsc{MadGraph} compared to 309 fb from AMI). Researching further, I was able to confirm that POWHEG includes $Z$/$\gamma$ interference in its $ZZ$ simulations, but the $W$ interference is not included because the effects have been shown to be negligible. 