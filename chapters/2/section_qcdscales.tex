\section{QCD scale uncertainties}
\label{sec:qcd}

My most recent contribution to the mono-$Z$ analysis has involved characterizing the systematic errors on the signal acceptance due to the uncertainties in the renormalization and factorization scales. These systematics are needed in order to calculate quantities such as signal and cross section upper limits at the later stages of the analysis. The renormalization scale $\mu_r$ arises from finite perturbation theory. QCD processes that are calculated beyond leading order generally lead to divergences in the amplitude integrals when the transferred momentum approaches infinity. There are methods to regularize these divergent integrals that result in a renormalization of the parameters of the theory. An arbitrary scale dependence arises from this process, which is known as the renormalization scale, $\mu_r$. The factorization scale $\mu_f$ is also a result of finite order perturbation theory. QCD is non-perturbative at large distances due to asymptotic freedom. Thus, in order to calculate meaningful quantities, such as cross sections, a factorization ansatz is introduced where the short-distance physics of the hard scatter are separated from the long-distance hadronic physics. This is illustrated in the following formula for the cross section of a hard scattering process for two hadrons $P_1$ and $P_2$:

\begin{equation}
\sigma = \sum_{i,j} \int \text{d}x_1 \text{d}x_2 f_{i} (x_1, \mu_f^2) f_{j} (x_2, \mu_f^2) \hat{\sigma}_{i,j} (x_1 P_1, x_2 P_2, \alpha_s(\mu_r), Q^2, \mu_r^2, \mu_f^2)
\end{equation}

\noindent Here $Q$ is the physical scale of the process, $x_1$ and $x_2$ are the momentum fractions of the interacting partons in $P_1$ and $P_2$, the functions $f_i$ and $f_j$ are the parton distribution functions for partons of types $i$ and $j$, and $\hat{\sigma}_{i,j}$ is the short-distance cross section for partons of types $i$ and $j$. $\hat{\sigma}_{i,j}$ can be calculated perturbatively because it only involves large momentum transfers where $\alpha_s$ is small, while the parton distribution functions contain the long-distance information of the overall interaction.

A truth-level study was performed in order to study the systematic uncertainties on the signal acceptance due to uncertainties in $\mu_r$ and $\mu_f$. ATHENA was used to run \textsc{MadGraph} and \textsc{Pythia} as one suite and generate truth DxAODs to be analyzed in ROOTCore. There are several truth derivations available. We chose to use the TRUTH1 derivation as it had a complete particle record as well as adequate thinning to avoid unmanageable file sizes.

The procedure used for the QCD scale uncertainties is as follows. For a given model, three signal samples are generated with modified scale factors according to $\mu_r = \mu_f = AQ$, where $A$ = 0.5, 1.0 (nominal), and 2.0. The samples are then put through a truth version of the mono-$Z$ ROOTCore analysis package, emulating all selections and cuts, as well as overlap removal. The systematic uncertainty is then calculated by observing the maximum percent difference in the signal yield compared to the nominal sample, after all stages in the cutflow. 

\begin{table}[!htb]
    \begin{subtable}{.5\linewidth}
      \centering
\begin{tabular}{|c||c|c|}
\hline
\multirow{2}{*}{$\frac{\mu_r}{Q}$, $\frac{\mu_f}{Q}$} & \multicolumn{2}{c|}{Yields}     \\ \cline{2-3} 
                              & $ee$           & $\mu\mu$       \\ \hline \hline
0.5                           & 40739          & 42603          \\ \hline
1.0                           & 39784          & 41470          \\ \hline
2.0                           & 39887          & 41000          \\ \hline \hline
$\delta$                      & 0.024 $\pm$ 0.007 & 0.027 $\pm$ 0.007 \\ \hline
\end{tabular}
\caption{$m_{\text{DM}}$ = 50 GeV, $m_{\text{med}}$ = 95 GeV.}
    \end{subtable}%
    \begin{subtable}{.5\linewidth}
      \centering
\begin{tabular}{|c||c|c|}
\hline
\multirow{2}{*}{$\frac{\mu_r}{Q}$, $\frac{\mu_f}{Q}$} & \multicolumn{2}{c|}{Yields}     \\ \cline{2-3} 
                              & $ee$           & $\mu\mu$       \\ \hline \hline
0.5                           & 82110          & 84195          \\ \hline
1.0                           & 82110          & 84403          \\ \hline
2.0                           & 81193          & 84230          \\ \hline \hline
$\delta$                      & 0.011 $\pm$ 0.005 & 0.003 $\pm$ 0.005 \\ \hline
\end{tabular}
\caption{$m_{\text{DM}}$ = 500 GeV, $m_{\text{med}}$ = 995 GeV.}
    \end{subtable} 
    \caption[QCD scale uncertainty results for vector-mediater dark matter samples]{QCD scale uncertainty results for two vector-mediater dark matter samples.}
\label{tbl:qcdresults}
\end{table}

The results given in Table \ref{tbl:qcdresults} are for two high-statistic (1M events), $s$-channel vector-mediator samples. The lower mass sample has errors on the order of 2-3\%, while the higher mass sample has errors on the order of 1\% and smaller. This difference is thought to be because of smaller PDF fluctuations at larger $Q$ values for the higher mass sample. We also see that the $\mu\mu$ signal region has a noticeably smaller error than for the $ee$ channel in the higher mass sample.
%%%% EXPLAIN %%%%

Other models and mass points are continuing to be studied as per the recommendations from the LHC Dark Matter Forum, with a focus on the vector-mediator and $t$-channel models. The samples studied so far typically have errors on the order of 1-3\%, which are comparable to the errors due to PDF uncertainties that we have also obtained recently, and are smaller than the current luminosity systematics of $\sim$ 5$\%$. In addition to studying more models and masses, we are also looking into characterizing uncertainties due to parton showering systematics. Once these have been assessed, they can be inputted into \texttt{HistFitter} to calculate upper limits on our signal.